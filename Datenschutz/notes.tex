\documentclass[12pt,a4paper]{article}
\usepackage{amsmath}
\usepackage{amssymb}

\title{Datenschutz}

\begin{document}
    \section{DSGVO \& Wichtige Begriffe}
    Bezieht zieht sich nur auf  (teil-)automatisierte Verarbeitung von personenbezogner Daten. \\
    \begin{itemize}
        \item \textbf{Personenbezogne Daten:}
            sind alle Angaben die sich auf eine identifizirte/idenifizierbare Person beziehen(z.B. Standort, Name, Adresse, Wohnverhältnis, Gehalt, Geburtsjahr, Kreditkartennummer, Telefonnummer)
        \item \textbf{Besondere Daten:}
            (z.B. Biometrie, Genetische Daten, Politsche Meinung, Gewerkschaftzugehörigkeit, Ethnische Herkunft, Weltanschauung, Gesundheit, Sexuelle Orientierung)
        \item \textbf{Pseudonymisierung/Anonymisierung/Aggregation}
            \begin{enumerate}
                \item Anonymisierung: das verändern von personenbezogner Datenm, sodass diese nicht oder nur mit unverhältnissmäßigen Auwand zu einer natürlichen Person zugeordnet werden kann.
                \item Pseudonymisierung: die Verarbeitung von Personenbezogenen Daten, so dass sie keiner Person zugeordnet werden kann. Die Information werden aber noch seperat abgespeichert.
                \item Aggregation: Treffen Aussagen über Gruppen(z.B. Durchschnitte)
            \end{enumerate}
        \item \textbf{Verarbeitung:}  \\
            Immer wenn eines der Sachen gemacht mit Daten gemacht wird:
            Erheben, Erfassen, Organisieren, Ordnen, Speichern, Anpassen oder Verändern, Auslesen, Abfragen, Verwenden, Offenlegung durch Übermittelung, Verbreitung oder andere Form der Bereitstellung, Abgleich oder die Verknüpfung, Einschränken, Löschen oder Vernichten
        \item \textbf{Verantwortlicher:}
            Die Person, die alleine oder gemeinsam über den Zweck und Mittel der Verarbeitun von personenbezogner Daten entscheidet.
        \item \textbf{''Marktortprinzip'':}
            Entweder ist der Verantwortliche in der EU oder die Verarbeitung bezieht sich auf Personen in der EU
        \item \textbf{Auftragsverarbeiter, Empfänger, Dritter}
            \begin{enumerate}
                \item Auftragsverarbeiter:
                    Die Person die im Auftrag des Verantwortlichen personenbezogne Daten verarbeitet.
                \item Empfänger:
                    - Wer kann daten lesen \\
                    - allen den personenbezogne Daten offengelgt werden
                \item Dritter:
                    alle die nicht teil der Persongruppen sind: betroffene Person, Verantwortlicher, Auftragsverarbeiter, und
                    beauftragte Personen
            \end{enumerate}
    \end{itemize}

    \section{Grundsätze}
    \begin{itemize}
        \item Rechtmäßigkeit, Verarbeitung nach Treu und Glauben, Transparenz
        \item Zweckbindung
        \item Datenminimierung
        \item Richtigkeit
        \item Speicherbegrenzung
        \item Integrität und Vertraulichkeit
    \end{itemize}
    Die Verarbeitung personenbezogener Daten ist nur erlaubt, wenn mindestens eine von sechs vorgegebenen Bedingungen erfüllt ist
    \begin{itemize}
        \item Einwilligung
        \item Vertragserfüllung oder vorvertraglich erforderlich
        \item Rechtliche Verpflichtung zur Verarbeitung
        \item Lebenswichtige Interessen des Betroffenen/Dritter
        \item Erforderlich für öffentliche Aufgabe
        \item Überwiegende berechtigte Interessen
    \end{itemize}
    \textbf{Transparenz}
    \begin{itemize}
        \item Daten sollen bei der betroffenen Person erhoben werden
        \item Vorrang der Direkterhebung
        \item Direkte Einsichtnahme in gespeicherte Daten ermöglichen (Betroffenenrechte)
        \item Trannsparenz schaffen:
            \begin{itemize}
                \item Was macht wer warum mit den Daten?
                \item Wann werden die gelöscht/anonymisiert?
            \end{itemize}
        \item Videoaufzeichnungen im öffentlichen Raum mit Beschilderung
    \end{itemize}
    \textbf{Zweckbindung}: 
        festgelegte, eindeutige und legitime Zwecke \\ \\
    \textbf{Datenminimierung}:
        nur für den Zweck notwendige Daten sind erlaubt \\ \\
    \textbf{Richtigkeit}:
        falsche Daten müssen, damit sie den Zweck dienen, sofort gelöscht bzw. korrigiert werden \\ \\
    \textbf{Speicherbegrenzung}:
        die Daten sollen nur solange gespeichert werden, wie der Zweck es braucht. \\ \\
    \textbf{Integrität und Vertraulichkeit:}
        man muss eine \textit{angemessene Sicherheit} gewerleisten.
        Dazu gehört Zugang von unrechtmäßigen Personen und technischen Problemn.\\ \\
    \textbf{Rechenschaftspflicht}:
        Die Verantwortlichen müssen die Einhaltung der oberenn Maßnahmen nachweisen können.
        \begin{itemize}
            \item Eintrag in das ''Verzeichnis der Verarbeitungstätigkeiten''
            \item Dokumentation über Datenschutzüberlegungen
            \item Dokumentation (Protokollierung) von Verarbeitungen
            \item Dokumentation einer Sicherung nach ''Stand der Technik''(Sicherheitskonzept)
            \item Protokolldaten an einzelnen Datensätzen
        \end{itemize}
    \section{Rechtsgrundlagen}
    Bedingungen für \textbf{Einwilligung:}\\
    \begin{itemize}
        \item freiwillig
        \item informiert(nichts in kleingedruckten)
        \item Bestimmt(eindeutiger Zewck)
        \item Wiederspruchmöglickeit(einfach)
        \item Nachweisepflicht der Einwilligung auf der Seite der Verantwortlichen
        \item Einwilligung bei Onlinediensten ab 16
    \end{itemize}
    Zusätzlich muss man Machstrukturen(z.B. Angestellter - Boss) betrachten. \\ \\
    Notwendig für die Erfüllung eines \textbf{Vertrages}\\
    Alle personenbezogenen Daten, der betroffenen Personen, dürfen so genutzt werden, sodass der Vertrag erfüllt werden kann. \\
    Das gilt auch schon bei Einholung von Angeboten(vorvertraglich) \\ \\
    \textbf{Rechtliche Pflichten:}\\
        z.B. Handels- und Steuerrecht verpflichten zur Aufbewahrung von Unterlagen, Arbeitsschutz, Kontaktdaten-Erfassung bei SARS-CoV-2
        \\ \\
    \textbf{Zum Schutz lebenswichtiger Interessen:}\\
        \begin{itemize}
            \item nur, wenn es keine Gesetzesgrundlage gibt.
            \item z.B. in Katastrophenfällen, Pandemien
            \item z.B. bewusstlose Personen
        \end{itemize}


\end{document}
